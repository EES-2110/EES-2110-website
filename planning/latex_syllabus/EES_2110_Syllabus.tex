\documentclass[11pt]{jgsyllabus}\usepackage[]{graphicx}\usepackage[]{xcolor}
% maxwidth is the original width if it is less than linewidth
% otherwise use linewidth (to make sure the graphics do not exceed the margin)
\makeatletter
\def\maxwidth{ %
  \ifdim\Gin@nat@width>\linewidth
    \linewidth
  \else
    \Gin@nat@width
  \fi
}
\makeatother

\definecolor{fgcolor}{rgb}{0.345, 0.345, 0.345}
\newcommand{\hlnum}[1]{\textcolor[rgb]{0.686,0.059,0.569}{#1}}%
\newcommand{\hlstr}[1]{\textcolor[rgb]{0.192,0.494,0.8}{#1}}%
\newcommand{\hlcom}[1]{\textcolor[rgb]{0.678,0.584,0.686}{\textit{#1}}}%
\newcommand{\hlopt}[1]{\textcolor[rgb]{0,0,0}{#1}}%
\newcommand{\hlstd}[1]{\textcolor[rgb]{0.345,0.345,0.345}{#1}}%
\newcommand{\hlkwa}[1]{\textcolor[rgb]{0.161,0.373,0.58}{\textbf{#1}}}%
\newcommand{\hlkwb}[1]{\textcolor[rgb]{0.69,0.353,0.396}{#1}}%
\newcommand{\hlkwc}[1]{\textcolor[rgb]{0.333,0.667,0.333}{#1}}%
\newcommand{\hlkwd}[1]{\textcolor[rgb]{0.737,0.353,0.396}{\textbf{#1}}}%
\let\hlipl\hlkwb

\usepackage{framed}
\makeatletter
\newenvironment{kframe}{%
 \def\at@end@of@kframe{}%
 \ifinner\ifhmode%
  \def\at@end@of@kframe{\end{minipage}}%
  \begin{minipage}{\columnwidth}%
 \fi\fi%
 \def\FrameCommand##1{\hskip\@totalleftmargin \hskip-\fboxsep
 \colorbox{shadecolor}{##1}\hskip-\fboxsep
     % There is no \\@totalrightmargin, so:
     \hskip-\linewidth \hskip-\@totalleftmargin \hskip\columnwidth}%
 \MakeFramed {\advance\hsize-\width
   \@totalleftmargin\z@ \linewidth\hsize
   \@setminipage}}%
 {\par\unskip\endMakeFramed%
 \at@end@of@kframe}
\makeatother

\definecolor{shadecolor}{rgb}{.97, .97, .97}
\definecolor{messagecolor}{rgb}{0, 0, 0}
\definecolor{warningcolor}{rgb}{1, 0, 1}
\definecolor{errorcolor}{rgb}{1, 0, 0}
\newenvironment{knitrout}{}{} % an empty environment to be redefined in TeX

\usepackage{alltt}
\usepackage{booktabs}
%\usepackage{tabularx}
\usepackage{multirow}

\setlength{\aboverulesep}{0.2ex}
\setlength{\belowrulesep}{0.2ex}
\setlength{\extrarowheight}{0.1ex}
\setlength{\heavyrulewidth}{1pt}
\setlength{\lightrulewidth}{0.05pt}

\newif\ifreading
\readingtrue

\iffalse
\newif{\ifisbn}
\isbnfalse

\newcommand{\shortdept}{EES}
\newcommand{\longdept}{EES}
\newcommand{\coursenum}{2110}
\newcommand{\sectionnum}{}
\newcommand{\shortcoursetitle}{Introduction to Climate Change}
\newcommand{\longcoursetitle}{\shortcoursetitle}
\newcommand{\semester}{Spring}
\newcommand{\yeartaught}{2023}

\newcommand{\Classroom}{Stevenson 6740}
\newcommand{\ClassMeetings}{MWF 9:05--9:55\ \Classroom}

\newcommand{\ProfName}{Jonathan Gilligan}
\newcommand{\ShortProfName}{J.\ Gilligan}
\newcommand{\ProfTitle}{Prof.\ Gilligan}
\newcommand{\ProfDescr}{The Alexander Heard Distinguished Service Professor,\\
Associate Professor of Earth \& Environmental Sciences,\\
Associate Professor of Civil \& Environmental Engineering}
\newcommand{\ProfOffice}{Stevenson 5735}
\newcommand{\ProfEmail}{\href{mailto:jonathan.gilligan@vanderbilt.edu}%
{\nolinkurl{jonathan.gilligan@vanderbilt.edu}}}
\newcommand{\ProfWeb}{\href{https://www.jonathangilligan.org}%
{\nolinkurl{www.jonathangilligan.org}}}
\newcommand{\ProfPhone}{}
\newcommand{\ProfOfficeHours}{Office Hours:
  Monday 10:05--11:00, Wednesday, 1:30--2:30,
	or by appointment.%
	} % TODO: Office Hours

\TAfalse
\TaMalefalse
\newcommand{\TaName}{Bryce Belanger}
\newcommand{\TaTitle}{Mr.\ Belanger}
\newcommand{\TaOffice}{Stevenson 5703A}
\newcommand{\TaOfficeLoc}{Office: \TaOffice}
\newcommand{\TaOfficeHours}{Office Hours:
Thurs.\ 9:00--11:00 AM
or by appointment.%
} % Tues. 2:30--3:30}
\TaPhonefalse
\ifTaPhone
\newcommand{\TaPhone}{\qqq}
\fi
\newcommand{\TaEmail}{\href{mailto:bryce.k.belanger@vanderbilt.edu}%
{\nolinkurl{bryce.k.belanger@vanderbilt.edu}}}
%{\href{qqq}%
%{\nolinkurl{qqq}}}

\newcommand{\FinalExamDay}{Thursday}   %TODO: Final Exam Date & Time
\newcommand{\ShortFinalExamDay}{Thurs.{}}   %TODO: Final Exam Date & Time
\newcommand{\FinalExamMonth}{May} %TODO: Final Exam Date & Time
\newcommand{\ShortFinalExamMonth}{May} %TODO: Final Exam Date & Time
\newcommand{\FinalExamDate}{4}  %TODO: Final Exam Date & Time
\newcommand{\FinalExamTime}{3:00--5:00~pm}  %TODO: Final Exam Date & Time
\newcommand{\FinalExamStartTime}{3:00~pm}
\newcommand{\FinalExamEndTime}{5:00~pm}
\newcommand{\FinalExamRoom}{\Classroom\ (our regular classroom)}  %TODO: Final Exam Date & Time

\newif\ifAltFinal
\AltFinalfalse

\ifAltFinal
\newcommand{\AltFinalExamDay}{Friday}   %TODO: Alternate Final Exam Date & Time
\newcommand{\ShortAltFinalExamDay}{Fri.{}}   %TODO: Alternate Final Exam Date & Time
\newcommand{\AltFinalExamMonth}{April} %TODO: Alternate Final Exam Date & Time
\newcommand{\ShortAltFinalExamMonth}{Apr.} %TODO: Final Exam Date & Time
\newcommand{\AltFinalExamDate}{28}  %TODO: Alternate Final Exam Date & Time
\newcommand{\AltFinalExamTime}{12:00--2:00~pm}  %TODO: Alternate Final Exam Date & Time
\newcommand{\AltFinalExamStartTime}{12:00~pm}
\newcommand{\AltFinalExamEndTime}{2:00~pm}
\newcommand{\AltFinalExamRoom}{\FinalExamRoom}  %TODO: Alternate Final Exam Date & Time
\fi

\newcommand{\TakeHomeExamDay}{\FinalExamDay}   %TODO: Final Exam Date & Time
\newcommand{\ShortTakeHomeExamDay}{\ShortFinalExamDay}   %TODO: Final Exam Date & Time
\newcommand{\TakeHomeExamMonth}{\FinalExamMonth} %TODO: Final Exam Date & Time
\newcommand{\ShortTakeHomeExamMonth}{\FinalExamMonth} %TODO: Final Exam Date & Time
\newcommand{\TakeHomeExamDate}{\FinalExamDate}  %TODO: Final Exam Date & Time
\newcommand{\TakeHomeExamTime}{\FinalExamDate}  %TODO: Final Exam Date & Time
\newcommand{\TakeHomeExamStartTime}{\FinalExamStartTime}
\newcommand{\TakeHomeExamEndTime}{\FinalExamEndTime}
\newcommand{\TakeHomeExamRoom}{\FinalExamRoom}  %TODO: Final Exam Date & Time


%\newcommand{\Ruddiman}{Earth's Climate}
%\newcommand{\LongRuddiman}{William F.\ Ruddiman,
%	\emph{Earth's Climate: Past and Future,} 2\textsuperscript{nd}~Ed.{}
%	(W.H. Freeman, 2008)%
%	}

\newcommand{\Archer}{Global Warming: Understanding the Forecast}
\newcommand{\ShortArcher}{Forecast}
\newcommand{\MedArcher}{Understanding the Forecast}
\newcommand{\LongArcher}{David Archer,
	\emph{Global Warming: Understanding the Forecast,} 2\textsuperscript{nd}~ed.{}
	(Wiley, 2011\ifisbn ; ISBN 978-0-470-94341-0\fi )%
	}
\newcommand{\ArcherURL}{\href{http://climatemodels.uchicago.edu/}%
{\nolinkurl{climatemodels.uchicago.edu}}}

\newcommand{\Forecast}{\Archer}
\newcommand{\ShortForecast}{\ShortArcher}
\newcommand{\MedForecast}{\MedArcher}
\newcommand{\LongForecast}{\LongArcher}
\newcommand{\ForecastURL}{\ArcherURL}

%\newcommand{\Houghton}{Global Warming: The Complete Briefing}
%\newcommand{\ShortHoughton}{Briefing}
%\newcommand{\MedHoughton}{Complete Briefing}
%\newcommand{\LongHoughton}{John Houghton,
%	\emph{Global Warming: The Complete Briefing}, 4\textsuperscript{th}~ed.{}
%	(Cambridge, 2009\ifisbn ; ISBN 978-0-521-70916-3\fi )%
%	}

%\newcommand{\Crisis}{Climate Crisis}
%\newcommand{\ShortCrisis}{Crisis}
%\newcommand{\LongCrisis}{David Archer \& Stefan Rahmstorf,
%	\emph{Climate Crisis: An Introductory Guide to Climate Change\/}
%	(Cambridge, 2010\ifisbn ; ISBN 978-0-521-73255-0\fi )%
%	}

\newcommand{\Pielke}{The Climate Fix}
\newcommand{\ShortPielke}{Climate Fix}
\newcommand{\MedPielke}{Climate Fix}
\newcommand{\LongPielke}{Roger A. Pielke, Jr.,
	\emph{The Climate Fix\/}
	(Basic Books, 2010\ifisbn; ISBN 978-0-465-02519-0\fi )%
	}
\newcommand{\Fix}{\Pielke}
\newcommand{\ShortFix}{\ShortPielke}
\newcommand{\MedFix}{\MedPielke}
\newcommand{\LongFix}{\LongPielke}
%\newcommand{\Ackerman}{Can We Afford the Future?}
%\newcommand{\ShortAckerman}{Future}
%\newcommand{\MedAckerman}{Can We Afford the Future?}
%\newcommand{\LongAckerman}{Frank Ackerman,
%	\emph{Can We Afford the Future? The Economics of a Warming World\/}
%	(Zed, 2009\ifisbn ;	ISBN 978-1-84813-038-8\fi )%
%	}

\newcommand{\Nordhaus}{The Climate Casino}
\newcommand{\ShortNordhaus}{Casino}
\newcommand{\MedNordhaus}{Climate Casino}
\newcommand{\LongNordhaus}{William Nordhaus,
    \emph{The Climate Casino: Risk, Uncertainty, and Economics for a Warming World\/}
    (Yale, 2013\ifisbn; ISBN 978-0-300-21264-8\fi)%
    }
\newcommand{\Casino}{\Nordhaus}
\newcommand{\ShortCasino}{\ShortNordhaus}
\newcommand{\MedCasino}{\MedNordhaus}
\newcommand{\LongCasino}{\LongNordhaus}

\newcommand{\Wickham}{R for Data Science}
\newcommand{\ShortWickham}{\Wickham}
\newcommand{\LongWickham}{Hadley Wickham and Garrett Grolemund,
  \emph{R for Data Science\/} (O'Reilly, 2017\ifisbn; ISBN 978-1-491-91039-9\fi)%
  }
\newcommand{\WickhamURL}{\href{http://r4ds.had.co.nz/}{\nolinkurl{r4ds.had.co.nz/}}}

\fi



\SetupHandout
\isbntrue

\renewcommand{\LongCourseName}{\ShortCourseName}%

\title{Syllabus\\
\ShortCourseNumber: \LongCourseName}

\fancyhead[L]{\bfseries\scshape \ShortCourseNumber\ Syllabus}

%
%
%

\IfFileExists{upquote.sty}{\usepackage{upquote}}{}
\begin{document}
\maketitle
\section[Nuts \& Bolts]{Nuts and Bolts}
\label{sec:NutsAndBolts}
\subsection{Class Meetings}
\ClassMeetings
\subsection{Professor}
\label{sec:Professor}
\ProfName\\
\ProfDescr\\
\ProfOffice\\
\ProfEmail\\
\ProfOfficeHours
%
%To book time during my office hours, send me an email
%or make an appointment on line via Google calendar at
%\url{http://bit.ly/o9ka5e}

\iffalse
\subsection{Teaching Assistant}
\label{sec:TA}
\TaName\\
\TaOfficeLoc\ifTaPhone , \TaPhone\fi\\
\TaEmail\\
\TaOfficeHours
\medskip

\noindent
\TaTitle\ will be grading homework, so address questions about your homework grades
to \TaAccusative.
\fi

\subsection{Email}
If you want to communicate with Professor Gilligan
\iffalse
or \TaTitle\
\fi
be sure to begin the
subject line of your email with ``{\scshape EES~2110}''. This helps assure
that I will see your message quickly and respond to it.

I have set my email reader to flag all messages like this as important, so I will read
them first. This also assures that I do not mistake your email for spam. I typically
receive over 100 emails per day, so if you do not follow these instructions I may not
notice your email.

\medskip
\clearpage
\section[Description]{Course Description}
\subsection{Catalog Description}

\textbf{EES 2110 Introduction to Climate Change}
Science, policy, history, and causes of climate change on Earth in the last
2 million years; evidence of human impacts on climate since 1850; future
climate change and its economic, social, and ecological consequences; economic,
technological, and public policy responses.
Not open to students who have earned credit for EES 3310 without permission.
[3] (MNS)

\subsection{Narrative Description}

This course will provide an introduction to the scientific principles of
earth's climate, the causes of climate change, and scientific knowledge about
the way climate has changed throughout our planet's history, with an emphasis
on the last two million years, and especially on recent history. We will study:
\begin{itemize}
  \item Determinants of climate: What factors affect climate, how do we know
    this, and how certain are we?
  \item Scientific evidence about past climates: What do we know, how do we know
    it, and how certain are we?
  \item Natural climate change in earth's history.
  \item Effects of human activity on global climate in the last 200 years.
  \item What do we know about future climate change and how will it affect the
    quality of people's lives?
  \item What can we do to mitigate future global climate change or adapt to life
    in a different climate?
  \item What is happening politically, both in the U.S. and internationally, to
    respond to climate change?
\end{itemize}

The course will make extensive use of algebra, but does not expect calculus or
advanced math.

\subsection{Goals for the Course}

My goals for this course are that at the end of the semester:
\begin{itemize}
\item You will have a solid quantitative understanding of the basic physical and
  chemical principles that control the system and be able to apply that
  knowledge to reasoning about the climate system and its response to
  disturbances.
\item You will have a solid scientific understanding of what scientists know, what
  they don't know, and how they know what they know about how climate works,
  how and why it has changed in the past, and how it may change in the future.
\item You will be able to evaluate the evidence for and against the idea that human
  activity is warming the planet and assess for yourself whether the evidence
  is persuasive.
\item You will have the tools and knowledge to make informed decisions about what
  climate policies you support or oppose.
\end{itemize}

When you leave this course, you will not be qualified to work as a climate
scientist, but you will be able to follow and critically evaluate news
reporting about climate change and climate policy, debate intelligently and
knowledgeably, and be an informed voter.

I do not care whether you agree with me politically. I respect people who think
for themselves.
\textbf{What counts is whether you can present your own position clearly and support
it with solid evidence and reasoned argument.}
%
%
%
%\clearpage
\section{Important Dates:}
Many of you have athletic and other commitments during the term and may travel for
personal reasons. As you plan for your semester, particularly if you are purchasing
nonrefundable airplane tickets, consult the syllabus.

If you have away games that require you to miss a scheduled test or in-class activity, let me know well in advance.
\begin{itemize}
\item There are in-class tests on
\textbf{Wednesday, February~8},
\textbf{Wednesday, March~8},
and
\textbf{Wednesday, April~12},
\item The take-home final exam is due at
\ifAltFinal
\textbf{\AltFinalExamEndTime~\AltFinalExamDay, \AltFinalExamMonth~\AltFinalExamDate}%
\else
\textbf{\FinalExamEndTime~\FinalExamDay, \FinalExamMonth~\FinalExamDate}%
\fi
.
You will submit the final exam on Brightspace, so you do not need to be on
campus for this.
\end{itemize}
%
%
%
%
%
%\clearpage
\section[Structure]{Structure of the Course:}
I divide the semester into two parts:
\begin{enumerate}
    \item \textbf{Scientific Principles of Climate}: For the first half of the semester,
    we will focus on the scientific principles of climate and natural climate change in earth's %'
    past. This will be very mathematical, using basic algebra. We do not use calculus or
    other advanced math in this class, but you should be comfortable with simple algebraic
    equations.
    We will then
    look at climate change in the last two centuries and what might
    happen over the next several centuries. We will emphasize examining the
    scientific evidence to understand what it can and cannot tell us.
    %
    \item \textbf{Human Dimensions of Climate Change:}
    For the
    second half of the semester, we will focus on the ways that climate change
    is likely to affect people's lives over the rest of this century and what
    technological responses are being considered to reduce harmful impacts.
    %'
\end{enumerate}
%
%
%
\subsection{Reading Material}
There are three required textbooks.
Supplementary reading on the Internet or in handouts will also be assigned during the term and posted on Brightspace.


\subsubsection[Required Reading]{Required Reading Materials}
\begin{itemize}
	%
	\item \LongArcher. Be sure you get the second edition because it is significantly different from the first.
	\item \LongNordhaus
	\item \LongPielke
\end{itemize}

{\sloppy
There is a companion web site to \MedArcher\ at \ArcherURL\
which includes interactive online computer models that we will use for
homework exercises.
You can also access that site at
\href{https://climatemodels.jgilligan.org}{\nolinkurl{climatemodels.jgilligan.org}}
}

\subsubsection{Overview of reading materials}
\MedArcher\ is an excellent introduction to
climte science, written by one of the leading climate scientists in the world
for students who are not science majors, but want to understand what science
knows about climate change.

Science aims to give correct answers to scientific questions, but there are
not right or wrong answers to questions of what is the best way to
assess the economic costs of climate change or the best policy with which to
respond to climate change, so I have chosen books and other reading material
that present different points of view on the political and economic aspects.

\subsection{Class Web Site}

In addition to the Brightspace web site, I have set up a server at
\href{https://ees2110.jgilligan.org}{\nolinkurl{ees2110.jgilligan.org}}
where I post the detailed reading and homework assignments,
slides from lectures, and other helpful material.


%\clearpage
\subsection{Graded Work}
%
%
%
\subsubsection[Grading]{Basis for Grading}

\begin{center}
	\begin{tabular}[t]{cr}
		Class participation \& in-class exercises & 5\%\\
		Homework & 30\%\\
		In-class tests & 45\%\\
		Final exam   & 20\%\\
	\end{tabular}
\end{center}
%
%
%
%
\subsection{Overview of reading assignments}

I have posted detailed reading assignments on the course web site, which give
specific pages to read for each class and notes on important things you should
understand.
\textbf{I expect you to complete the reading before you come to class on the day
for which the reading is assigned},
so you can participate in discussions of the
assigned material and ask questions if there are things you don't understand.
%'

\subsection{Homework Assignments}

I have posted homework assignments on the course web site. Turn in homework
assignments on Brightspace. Homework assignments are due at the time specified
in the Brightspace assignment, which is usually at the start of class on the
day they are due.

Late homework will be penalized 5\% for each day, or part of a day, it is late
until I post the solutions (usually a week after it is due, but sooner if a
test is coming up).
After I have posted the solutions, late homework will receive up to 50\% credit,
if it is turned in by April 21, so no matter how late it is, it will still
be worth finishing.


\subsection{Tests and Exams}

There will be three tests during the semester and a take-home final exam.
The in-class tests will test you on the material we have studied. The first
two will cover the science and the third will cover responses to climate change.

I will provide you with a list of relevant physical constants and important
equations. For the most part, I do not expect you to memorize equations and
numbers, but there are a few numbers, such as the equilibrium climate
sensitivity and the concentration of carbon dioxide in the atmosphere that
I do expect you to know by heart. Reading assignments will clearly state what
numbers I expect you to memorize.

The final exam will be an open-book, open-notes take-home exam.
It will be cumulative over all the material covered during the term.
It will not ask you to solve scientific problems, but will
ask open-ended questions about the big ideas we have covered during the semester.

The in-class tests will take place on
\textbf{Wednesday, February~8},
\textbf{Wednesday, March~8},
and
\textbf{Wednesday, April~12}.

The take-home final exam is due at
\ifAltFinal
\textbf{\AltFinalExamEndTime~\AltFinalExamDay, \AltFinalExamMonth~\AltFinalExamDate}%
\else
\textbf{\FinalExamEndTime~\FinalExamDay, \FinalExamMonth~\FinalExamDate}%
\fi
.

%\clearpage
\section{Honor Code:}
This course, like all courses at Vanderbilt, is conducted under the Honor Code.

I encourage you to to seek help from me, from \TaTitle,
or from other classmates or friends in your studying. I also encourage working together
on homework assignments: you may talk with your friends and classmates about
homework assignments, compare notes on how you are working a problem, and you
may look at your classmates' work on homework assignments. %'
But you must work through the problems yourself
in the work you turn in: \textbf{Even if you have discussed the solution with others you must
work through the steps yourself and express the answers in your own words. You may not
simply copy someone else's answer.} %'

Tests are different from homework: \textbf{%
%all work you turn in and
all work on tests and exams must be entirely your own}.
\textbf{You may not work together with anyone or receive any help from anyone but me
% or \TaTitle\
on exams and tests (this includes take-home exams and tests)}.

If you ever
have questions about how the Honor Code applies to your work
in this course, please ask me.
\textbf{Uncertainty about the Honor Code does not excuse a violation.}

\section{Final Note:}

I have made every effort to plan a busy, exciting, and instructive semester.
I may find during the term that I need to revise the syllabus to give more time
to some subjects or to pass more quickly over others rather than covering them
in depth. Many topics we will cover are frequently in the news. Breaking news
may warrant a detour from the schedule presented on the following pages.
Thus, while I will attempt to follow this syllabus as closely as I can,
you should realize that it is subject to change during the semester.
%
%
\iftrue
% \pagebreak
\section{Meet Your Professor}
Jonathan Gilligan has worked in many areas of science and public policy.
His past research includes work on laser physics, quantum optics,
laser surgery, electrical properties of the heart, using modified spy planes to
study the ozone layer in the stratosphere, and connections between religion and
care for the environment.
\iffalse

Professor Gilligan joined the Vanderbilt Faculty in 1994 as a member of the
Department of Physics and Astronomy. In 2003, when the Department of Geology
became the Department of Earth and Environmental Science, Professor Gilligan
joined the new department to focus on atmospheric science, global climate change,
and the interactions of politics, ethics, religion, communication, and the
environment.
\fi

Professor Gilligan is
the Alexander Heard Distinguished Service Professor,
Associate Professor of Earth \& Environmental Sciences,
Associate Professor of Civil \& Environmental Engineering, and
the director of the Vanderbilt Climate and Society Grand Challenge Initiative,
which is working to integrate research, teaching, and public outreach about
climate change across the natural sciences, social sciences, and humanities.
He co-chaired the committee that designed the new Climate Studies major at
Vanderbilt and he is currently serving on the faculty committee that is
redesigning Vanderbilt's undergraduate Arts and Science curriculum.
%'

His current research investigates the role of individual and household behavior
in greenhouse gas emissions in the United States;
how ``smart cities'' can use technology to reduce environmental footprints and
promote health and citizen empowerment;
water conservation policies in American cities;
vulnerability and resilience to environmental stress in South Asia;
and developing new directions for climate policy in the US.

Professor Gilligan and Professor Michael Vandenbergh won
the 2017 Morrison Prize for the highest-impact paper of the year
on sustainability law and policy.
Gilligan and Vandenbergh's book, %'
\emph{Beyond Politics: The Private Governance Approach to Climate Change\/}
(Cambridge University Press, 2017),
was named by \emph{Environmental Forum\/} as one of the most
important books on the environment of the last 50 years.

Apart from his academic work, Professor Gilligan dabbles in writing for the
theater. His stage adaptation of Nathaniel Hawthorne's %'
\emph{The Scarlet Letter},
co-written with his mother Carol Gilligan, has been staged at The Culture
Project in New York City, starring
Marisa Tomei, Ron Cephas Jones, and Bobby Cannavale, and was later performed
at Prime Stage Theatre, Pittsburgh and in a touring production by The National
Players. Most recently, it was performed as the principal fall 2019 production
of the Fullerton College Classic Dramatic Series in Fullerton CA,
directed by Michael Mueller,
and was also chosen by the Classic Repertory Company in Watertown, MA,
for its 2019--2020 repertory season.

Prof.\ Gilligan and Carol Gilligan also wrote the libretto for an opera,
\emph{Pearl}, in collaboration composer Amy Scurria, and producer/conductor
Sara Jobin, which was performed at Shakespeare \& Company in Lenox MA,
starring Maureen O'Flynn, John Bellemer, Marnie Breckenridge, John Cheek, %'
and Michael Corvino, and in Shanghai China,
% as part of a cultural exchange,
starring Li Xin, Wang Yang, John Bellemer, and Lin Shu.
\fi
%
%
%
%
%
%
%\end{document}

%\cleardoublepage
\clearpage
\appendix
\setcounter{secnumdepth}{0}
\newcommand{\maybehline}{\hline}%
\setlength\extrarowheight{4pt}
\section[Class Schedule]{Schedule of Classes
\ifrevised
	(Revised \RevisionDate)%
\else
	(Subject to Change)%
\fi}

\textbf{\scshape Important Note:} This schedule gives a rough indication of the reading for each day. See the assignment sheets posted on Brightspace for the detailed daily assignments.

\setlength{\aboverulesep}{0.2ex}
\setlength{\belowrulesep}{0.2ex}
\setlength{\extrarowheight}{0.1ex}
\setlength{\heavyrulewidth}{0.5pt}
\setlength{\lightrulewidth}{0.05pt}
\begin{center}
% latex table generated in R 4.2.2 by xtable 1.8-4 package
% Mon Jan  9 01:43:24 2023
\begin{tabular}{l@{~}c@{~}r>{\raggedright}m{2.6in}>{\centering}m{2in}c}
  \toprule
  \multicolumn{3}{c}{\bfseries Date} & \multicolumn{1}{c}{\bfseries Topic} &\multicolumn{1}{c}{\bfseries Reading} &\\
 \midrule
Mon., & Jan. &   9 & Introduction & No reading &  \\ 
   \midrule
Wed., & Jan. &  11 & What is Climate Change? & \emph{\ShortArcher\/} Ch.~1, \emph{\ShortNordhaus\/} Ch.~1--2 &  \\ 
   \midrule
Fri., & Jan. &  13 & Energy Balance and Climate & \emph{\ShortArcher\/} Ch.~2--3 pp.~9--23 &  \\ 
   \midrule
Mon., & Jan. &  16 & \multicolumn{2}{l}{\bfseries\scshape\large Martin Luther King, Jr. Day} & \\% & No reading &  \\ 
   \midrule
Wed., & Jan. &  18 & Greenhouse Effect & \emph{\ShortArcher\/} Ch.~3 pp.~23--26 &  \\ 
   \midrule
Fri., & Jan. &  20 & Greenhouse Gases & \emph{\ShortArcher\/} Ch.~4 Read the whole chapter, but concentrate on pp.~29-34 &  \\ 
   \midrule
Mon., & Jan. &  23 & Spectrum of Atmospheric Radiation & \emph{\ShortArcher\/} Ch.~4 pp.~34-40 &  \\ 
   \midrule
Wed., & Jan. &  25 & Vertical Structure of the Atmosphere & \emph{\ShortArcher\/} Ch.~5 Read the whole chapter, but focus on pp.~43-50 &  \\ 
   \midrule
Fri., & Jan. &  27 & Atmospheric Convection & \emph{\ShortArcher\/} Ch.~5 pp.~50-55 &  \\ 
   \midrule
Mon., & Jan. &  30 & Review of Greenhouse Effect & No reading &  \\ 
   \midrule
Wed., & Feb. &   1 & Weather and Climate & \emph{\ShortArcher\/} 6 &  \\ 
   \midrule
Fri., & Feb. &   3 & Feedbacks & \emph{\ShortArcher\/} 7 pp.~73--81 &  \\ 
   \midrule
Mon., & Feb. &   6 & Ocean and Biosphere Feedbacks & \emph{\ShortArcher\/} 7 81--84 &  \\ 
   \midrule
Wed., & Feb. &   8 & \multicolumn{2}{l}{\bfseries\scshape\large Test \#1} & \\% & No reading &  \\ 
   \midrule
Fri., & Feb. &  10 & The Carbon Cycle: Ocean and Biosphere & \emph{\ShortArcher\/} 8 pp.~89--97 &  \\ 
   \midrule
Mon., & Feb. &  13 & The Carbon Cycle: Mineral Weathering & \emph{\ShortArcher\/} 8 pp.~95--101 &  \\ 
   \midrule
Wed., & Feb. &  15 & Perturbing the Carbon Cycle & \emph{\ShortArcher\/} 10 &  \\ 
   \midrule
Fri., & Feb. &  17 & Ocean Acidification & \emph{\ShortArcher\/} 10 &  \\ 
   \midrule
Mon., & Feb. &  20 & Isotopes & Handouts (on Brightspace) &  \\ 
   \midrule
Wed., & Feb. &  22 & Climates of the Past & \emph{\ShortArcher\/} 11 &  \\ 
   \midrule
Fri., & Feb. &  24 & The Pleistocene Ice Ages & \emph{\ShortArcher\/} 7, \emph{\ShortArcher\/} 8, \emph{\ShortArcher\/} 11 &  \\ 
   \midrule
Mon., & Feb. &  27 & Human Impact on Climate: The Evidence & \emph{\ShortArcher\/} 11 &  \\ 
   \midrule
Wed., & Mar. &   1 & Review of Carbon Cycle and Paleoclimates & No reading &  \\ 
   \bottomrule
\end{tabular}

\end{center}

\begin{center}
% latex table generated in R 4.2.2 by xtable 1.8-4 package
% Mon Jan  9 01:43:24 2023
\begin{tabular}{l@{~}c@{~}r>{\raggedright}m{2.6in}>{\centering}m{2in}c}
  \toprule
  \multicolumn{3}{c}{\bfseries Date} & \multicolumn{1}{c}{\bfseries Topic} &\multicolumn{1}{c}{\bfseries Reading} &\\
 \midrule
Fri., & Mar. &   3 & Identifying Fallacies about Climate Change & ``23 Ways to Mislead'' &  \\ 
   \midrule
Mon., & Mar. &   6 & Climate Fallacies Game & ``A History of FLICC'' &  \\ 
   \midrule
Wed., & Mar. &   8 & \multicolumn{2}{l}{\bfseries\scshape\large Test \#2} & \\% & No reading &  \\ 
   \midrule
Fri., & Mar. &  10 & Climate Models & \emph{\ShortNordhaus\/} Ch.~3--4, \emph{\ShortNordhaus\/} 3--4 &  \\ 
   \midrule
Mon., & Mar. &  13 & \multicolumn{2}{l}{\multirow{3}{*}{\bfseries\scshape\Large Spring Break}} & \\% &  &  \\ 
  Wed., & Mar. &  15 &  &  &  \\ 
  Fri., & Mar. &  17 &  &  &  \\ 
   \midrule
Mon., & Mar. &  20 & Future Climate Change & \emph{\ShortArcher\/} Ch.~12 pp.~153--164, \emph{\ShortNordhaus\/} Ch.~5 &  \\ 
   \midrule
Wed., & Mar. &  22 & Uncertainty about Future Climates & \emph{\ShortArcher\/} Ch.~12 pp.~164--166, \emph{\ShortNordhaus\/} Ch.~24, \emph{\ShortPielke\/} Ch.~1 pp.~1--24 &  \\ 
   \midrule
Fri., & Mar. &  24 & Policy Myths & \emph{\ShortPielke\/} 2, \emph{\ShortNordhaus\/} Ch.~25 &  \\ 
   \midrule
Mon., & Mar. &  27 & The Kaya Identity & \emph{\ShortPielke\/} Ch.~3, \emph{\ShortNordhaus\/} Ch.~14 &  \\ 
   \midrule
Wed., & Mar. &  29 & Future Emission Scenarios & TBA &  \\ 
   \midrule
Fri., & Mar. &  31 & Renewable Energy & Handouts (on Brightspace) &  \\ 
   \midrule
Mon., & Apr. &   3 & Nuclear Energy & Handouts (on Brightspace) &  \\ 
   \midrule
Wed., & Apr. &   5 & Geoengineering: Solar Radiation Management & TBA &  \\ 
   \midrule
Fri., & Apr. &   7 & Geoengineering: Carbon Dioxide Management & TBA &  \\ 
   \midrule
Mon., & Apr. &  10 & Review of Impacts and Responses to Climate Change & No reading &  \\ 
   \midrule
Wed., & Apr. &  12 & \multicolumn{2}{l}{\bfseries\scshape\large Test \#3} & \\% & No reading &  \\ 
   \midrule
Fri., & Apr. &  14 & Climate Change, Agriculture, and Health & \emph{\ShortNordhaus\/} 6--8 &  \\ 
   \midrule
Mon., & Apr. &  17 & Climate and Oceans: Hurricanes, Sea-level, etc. & \emph{\ShortNordhaus\/} 9--10, 12 &  \\ 
   \midrule
Wed., & Apr. &  19 & Climate Justice & Handouts (on Brightspace) &  \\ 
   \midrule
Fri., & Apr. &  21 & History of Climate Policy & TBA &  \\ 
   \midrule
Mon., & Apr. &  24 & Future Possibilities & TBA &  \\ 
   \midrule
Thu., & May. &   4 & \multicolumn{2}{l}{\bfseries\scshape\large Take-home final exam due} & \\% & No reading &  \\ 
   \bottomrule
\end{tabular}

\end{center}
%
\end{document}
